% Abstract

%\renewcommand{\abstractname}{Abstract} % Uncomment to change the name of the abstract

\pdfbookmark[1]{Abstract [Spanish]}{Abstract [Spanish]} % Bookmark name visible in a PDF viewer

\begingroup
\let\clearpage\relax
\let\cleardoublepage\relax
\let\cleardoublepage\relax

\chapter*{Resumen}
Los \textbf{veh\'iculos auton\'omos} ser\'an parte de la realidad de las ciudades en un futuro no muy lejano. La mayor parte de la atenci\'on est\'a fijada en los coches, aunque deber\'an superar una gran cantidad obst\'aculos antes de ocupar las urbes, dejando un espacio de crecimiento para otras alternativas. Esta tesis es sobre una de esas alternativas, el \textbf{Persuasive Electric Vehicle}, y las metodolog\'ias implementadas para superar dichos obst\'aculos. Especialmente interesante es el problema de la \textbf{Localizaci\'on Simult\'anea y Mapeado} debido a su complejidad y las posibilidades que abre para el desarrollo de plataformas completamente auton\'onomas. En los siguientes cap\'itulos, el lector aprender\'a como se ha afrontado dicho problema, cu\'ales son los resultados empleando diferentes t\'ecnicas y qu\'e aspectos se deber\'ia reforzar para ver en el futuro veh\'iculos aut\'onomos ligeros circular por las ciudades del mundo. 

\keywordsesp{Veh\'iculos Aut\'onomos, Ciudades, Plataformas Ligeras, Movilidad compartida, Localizaci\'on Simult\'anea y Mapeado}

\endgroup			

\vfill